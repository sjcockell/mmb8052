%%
% Copyright (c) 2017 - 2021, Pascal Wagler;
% Copyright (c) 2014 - 2021, John MacFarlane
%
% All rights reserved.
%
% Redistribution and use in source and binary forms, with or without
% modification, are permitted provided that the following conditions
% are met:
%
% - Redistributions of source code must retain the above copyright
% notice, this list of conditions and the following disclaimer.
%
% - Redistributions in binary form must reproduce the above copyright
% notice, this list of conditions and the following disclaimer in the
% documentation and/or other materials provided with the distribution.
%
% - Neither the name of John MacFarlane nor the names of other
% contributors may be used to endorse or promote products derived
% from this software without specific prior written permission.
%
% THIS SOFTWARE IS PROVIDED BY THE COPYRIGHT HOLDERS AND CONTRIBUTORS
% "AS IS" AND ANY EXPRESS OR IMPLIED WARRANTIES, INCLUDING, BUT NOT
% LIMITED TO, THE IMPLIED WARRANTIES OF MERCHANTABILITY AND FITNESS
% FOR A PARTICULAR PURPOSE ARE DISCLAIMED. IN NO EVENT SHALL THE
% COPYRIGHT OWNER OR CONTRIBUTORS BE LIABLE FOR ANY DIRECT, INDIRECT,
% INCIDENTAL, SPECIAL, EXEMPLARY, OR CONSEQUENTIAL DAMAGES (INCLUDING,
% BUT NOT LIMITED TO, PROCUREMENT OF SUBSTITUTE GOODS OR SERVICES;
% LOSS OF USE, DATA, OR PROFITS; OR BUSINESS INTERRUPTION) HOWEVER
% CAUSED AND ON ANY THEORY OF LIABILITY, WHETHER IN CONTRACT, STRICT
% LIABILITY, OR TORT (INCLUDING NEGLIGENCE OR OTHERWISE) ARISING IN
% ANY WAY OUT OF THE USE OF THIS SOFTWARE, EVEN IF ADVISED OF THE
% POSSIBILITY OF SUCH DAMAGE.
%%

%%
% This is the Eisvogel pandoc LaTeX template.
%
% For usage information and examples visit the official GitHub page:
% https://github.com/Wandmalfarbe/pandoc-latex-template
%%

% Options for packages loaded elsewhere
\PassOptionsToPackage{unicode}{hyperref}
\PassOptionsToPackage{hyphens}{url}
\PassOptionsToPackage{dvipsnames,svgnames,x11names,table}{xcolor}
%
\documentclass[
  english,
  paper=a4,
  ,captions=tableheading
]{scrartcl}
\usepackage{amsmath,amssymb}
\usepackage{lmodern}
\usepackage{setspace}
\setstretch{1.2}
\usepackage{iftex}
\ifPDFTeX
  \usepackage[T1]{fontenc}
  \usepackage[utf8]{inputenc}
  \usepackage{textcomp} % provide euro and other symbols
\else % if luatex or xetex
  \usepackage{unicode-math}
  \defaultfontfeatures{Scale=MatchLowercase}
  \defaultfontfeatures[\rmfamily]{Ligatures=TeX,Scale=1}
\fi
% Use upquote if available, for straight quotes in verbatim environments
\IfFileExists{upquote.sty}{\usepackage{upquote}}{}
\IfFileExists{microtype.sty}{% use microtype if available
  \usepackage[]{microtype}
  \UseMicrotypeSet[protrusion]{basicmath} % disable protrusion for tt fonts
}{}
\makeatletter
\@ifundefined{KOMAClassName}{% if non-KOMA class
  \IfFileExists{parskip.sty}{%
    \usepackage{parskip}
  }{% else
    \setlength{\parindent}{0pt}
    \setlength{\parskip}{6pt plus 2pt minus 1pt}}
}{% if KOMA class
  \KOMAoptions{parskip=half}}
\makeatother
\usepackage{xcolor}
\definecolor{default-linkcolor}{HTML}{5f9baf}
\definecolor{default-filecolor}{HTML}{A50000}
\definecolor{default-citecolor}{HTML}{4077C0}
\definecolor{default-urlcolor}{HTML}{5f9baf}
\IfFileExists{xurl.sty}{\usepackage{xurl}}{} % add URL line breaks if available
\IfFileExists{bookmark.sty}{\usepackage{bookmark}}{\usepackage{hyperref}}
\hypersetup{
  pdftitle={MMB8052 - Practical 1},
  pdfauthor={Simon J Cockell},
  pdflang={en},
  pdfsubject={Bioinformatics Practical},
  pdfkeywords={Bioinformatics, Linux},
  hidelinks,
  breaklinks=true,
  pdfcreator={LaTeX via pandoc with the Eisvogel template}}
\urlstyle{same} % disable monospaced font for URLs
\usepackage[margin=2.5cm,includehead=true,includefoot=true,centering,]{geometry}
\usepackage{color}
\usepackage{fancyvrb}
\newcommand{\VerbBar}{|}
\newcommand{\VERB}{\Verb[commandchars=\\\{\}]}
\DefineVerbatimEnvironment{Highlighting}{Verbatim}{commandchars=\\\{\}}
% Add ',fontsize=\small' for more characters per line
\usepackage{framed}
\definecolor{shadecolor}{RGB}{248,248,248}
\newenvironment{Shaded}{\begin{snugshade}}{\end{snugshade}}
\newcommand{\AlertTok}[1]{\textcolor[rgb]{0.94,0.16,0.16}{#1}}
\newcommand{\AnnotationTok}[1]{\textcolor[rgb]{0.56,0.35,0.01}{\textbf{\textit{#1}}}}
\newcommand{\AttributeTok}[1]{\textcolor[rgb]{0.77,0.63,0.00}{#1}}
\newcommand{\BaseNTok}[1]{\textcolor[rgb]{0.00,0.00,0.81}{#1}}
\newcommand{\BuiltInTok}[1]{#1}
\newcommand{\CharTok}[1]{\textcolor[rgb]{0.31,0.60,0.02}{#1}}
\newcommand{\CommentTok}[1]{\textcolor[rgb]{0.56,0.35,0.01}{\textit{#1}}}
\newcommand{\CommentVarTok}[1]{\textcolor[rgb]{0.56,0.35,0.01}{\textbf{\textit{#1}}}}
\newcommand{\ConstantTok}[1]{\textcolor[rgb]{0.00,0.00,0.00}{#1}}
\newcommand{\ControlFlowTok}[1]{\textcolor[rgb]{0.13,0.29,0.53}{\textbf{#1}}}
\newcommand{\DataTypeTok}[1]{\textcolor[rgb]{0.13,0.29,0.53}{#1}}
\newcommand{\DecValTok}[1]{\textcolor[rgb]{0.00,0.00,0.81}{#1}}
\newcommand{\DocumentationTok}[1]{\textcolor[rgb]{0.56,0.35,0.01}{\textbf{\textit{#1}}}}
\newcommand{\ErrorTok}[1]{\textcolor[rgb]{0.64,0.00,0.00}{\textbf{#1}}}
\newcommand{\ExtensionTok}[1]{#1}
\newcommand{\FloatTok}[1]{\textcolor[rgb]{0.00,0.00,0.81}{#1}}
\newcommand{\FunctionTok}[1]{\textcolor[rgb]{0.00,0.00,0.00}{#1}}
\newcommand{\ImportTok}[1]{#1}
\newcommand{\InformationTok}[1]{\textcolor[rgb]{0.56,0.35,0.01}{\textbf{\textit{#1}}}}
\newcommand{\KeywordTok}[1]{\textcolor[rgb]{0.13,0.29,0.53}{\textbf{#1}}}
\newcommand{\NormalTok}[1]{#1}
\newcommand{\OperatorTok}[1]{\textcolor[rgb]{0.81,0.36,0.00}{\textbf{#1}}}
\newcommand{\OtherTok}[1]{\textcolor[rgb]{0.56,0.35,0.01}{#1}}
\newcommand{\PreprocessorTok}[1]{\textcolor[rgb]{0.56,0.35,0.01}{\textit{#1}}}
\newcommand{\RegionMarkerTok}[1]{#1}
\newcommand{\SpecialCharTok}[1]{\textcolor[rgb]{0.00,0.00,0.00}{#1}}
\newcommand{\SpecialStringTok}[1]{\textcolor[rgb]{0.31,0.60,0.02}{#1}}
\newcommand{\StringTok}[1]{\textcolor[rgb]{0.31,0.60,0.02}{#1}}
\newcommand{\VariableTok}[1]{\textcolor[rgb]{0.00,0.00,0.00}{#1}}
\newcommand{\VerbatimStringTok}[1]{\textcolor[rgb]{0.31,0.60,0.02}{#1}}
\newcommand{\WarningTok}[1]{\textcolor[rgb]{0.56,0.35,0.01}{\textbf{\textit{#1}}}}

% Workaround/bugfix from jannick0.
% See https://github.com/jgm/pandoc/issues/4302#issuecomment-360669013)
% or https://github.com/Wandmalfarbe/pandoc-latex-template/issues/2
%
% Redefine the verbatim environment 'Highlighting' to break long lines (with
% the help of fvextra). Redefinition is necessary because it is unlikely that
% pandoc includes fvextra in the default template.
\usepackage{fvextra}
\DefineVerbatimEnvironment{Highlighting}{Verbatim}{breaklines,fontsize=\small,commandchars=\\\{\}}

\usepackage{longtable,booktabs,array}
\usepackage{calc} % for calculating minipage widths
% Correct order of tables after \paragraph or \subparagraph
\usepackage{etoolbox}
\makeatletter
\patchcmd\longtable{\par}{\if@noskipsec\mbox{}\fi\par}{}{}
\makeatother
% Allow footnotes in longtable head/foot
\IfFileExists{footnotehyper.sty}{\usepackage{footnotehyper}}{\usepackage{footnote}}
\makesavenoteenv{longtable}
% add backlinks to footnote references, cf. https://tex.stackexchange.com/questions/302266/make-footnote-clickable-both-ways
\usepackage{footnotebackref}
\usepackage{graphicx}
\makeatletter
\def\maxwidth{\ifdim\Gin@nat@width>\linewidth\linewidth\else\Gin@nat@width\fi}
\def\maxheight{\ifdim\Gin@nat@height>\textheight\textheight\else\Gin@nat@height\fi}
\makeatother
% Scale images if necessary, so that they will not overflow the page
% margins by default, and it is still possible to overwrite the defaults
% using explicit options in \includegraphics[width, height, ...]{}
\setkeys{Gin}{width=\maxwidth,height=\maxheight,keepaspectratio}
% Set default figure placement to htbp
\makeatletter
\def\fps@figure{htbp}
\makeatother
\setlength{\emergencystretch}{3em} % prevent overfull lines
\providecommand{\tightlist}{%
  \setlength{\itemsep}{0pt}\setlength{\parskip}{0pt}}
\setcounter{secnumdepth}{-\maxdimen} % remove section numbering
\usepackage[main=english]{babel}
% get rid of language-specific shorthands (see #6817):
\let\LanguageShortHands\languageshorthands
\def\languageshorthands#1{}
\ifLuaTeX
  \usepackage{selnolig}  % disable illegal ligatures
\fi

\title{MMB8052 - Practical 1}
\usepackage{etoolbox}
\makeatletter
\providecommand{\subtitle}[1]{% add subtitle to \maketitle
  \apptocmd{\@title}{\par {\large #1 \par}}{}{}
}
\makeatother
\subtitle{Introduction to the Linux Command Line}
\author{Simon J Cockell}
\date{}



%%
%% added
%%


%
% for the background color of the title page
%
\usepackage{pagecolor}
\usepackage{afterpage}
\usepackage{tikz}
\usepackage[margin=2.5cm,includehead=true,includefoot=true,centering]{geometry}

%
% break urls
%
\PassOptionsToPackage{hyphens}{url}

%
% When using babel or polyglossia with biblatex, loading csquotes is recommended
% to ensure that quoted texts are typeset according to the rules of your main language.
%
\usepackage{csquotes}

%
% captions
%
\definecolor{caption-color}{HTML}{777777}
\usepackage[font={stretch=1.2}, textfont={color=caption-color}, position=top, skip=4mm, labelfont=bf, singlelinecheck=false, justification=raggedright]{caption}
\setcapindent{0em}

%
% blockquote
%
\definecolor{blockquote-border}{RGB}{221,221,221}
\definecolor{blockquote-text}{RGB}{119,119,119}
\usepackage{mdframed}
\newmdenv[rightline=false,bottomline=false,topline=false,linewidth=3pt,linecolor=blockquote-border,skipabove=\parskip]{customblockquote}
\renewenvironment{quote}{\begin{customblockquote}\list{}{\rightmargin=0em\leftmargin=0em}%
\item\relax\color{blockquote-text}\ignorespaces}{\unskip\unskip\endlist\end{customblockquote}}

%
% Source Sans Pro as the de­fault font fam­ily
% Source Code Pro for monospace text
%
% 'default' option sets the default
% font family to Source Sans Pro, not \sfdefault.
%
\ifnum 0\ifxetex 1\fi\ifluatex 1\fi=0 % if pdftex
    \usepackage[default]{sourcesanspro}
  \usepackage{sourcecodepro}
  \else % if not pdftex
    \usepackage[default]{sourcesanspro}
  \usepackage{sourcecodepro}

  % XeLaTeX specific adjustments for straight quotes: https://tex.stackexchange.com/a/354887
  % This issue is already fixed (see https://github.com/silkeh/latex-sourcecodepro/pull/5) but the
  % fix is still unreleased.
  % TODO: Remove this workaround when the new version of sourcecodepro is released on CTAN.
  \ifxetex
    \makeatletter
    \defaultfontfeatures[\ttfamily]
      { Numbers   = \sourcecodepro@figurestyle,
        Scale     = \SourceCodePro@scale,
        Extension = .otf }
    \setmonofont
      [ UprightFont    = *-\sourcecodepro@regstyle,
        ItalicFont     = *-\sourcecodepro@regstyle It,
        BoldFont       = *-\sourcecodepro@boldstyle,
        BoldItalicFont = *-\sourcecodepro@boldstyle It ]
      {SourceCodePro}
    \makeatother
  \fi
  \fi

%
% heading color
%
\definecolor{heading-color}{RGB}{40,40,40}
\addtokomafont{section}{\color{heading-color}}
% When using the classes report, scrreprt, book,
% scrbook or memoir, uncomment the following line.
%\addtokomafont{chapter}{\color{heading-color}}

%
% variables for title, author and date
%
\usepackage{titling}
\title{MMB8052 - Practical 1}
\author{Simon J Cockell}
\date{}

%
% tables
%

\definecolor{table-row-color}{HTML}{F5F5F5}
\definecolor{table-rule-color}{HTML}{999999}

%\arrayrulecolor{black!40}
\arrayrulecolor{table-rule-color}     % color of \toprule, \midrule, \bottomrule
\setlength\heavyrulewidth{0.3ex}      % thickness of \toprule, \bottomrule
\renewcommand{\arraystretch}{1.3}     % spacing (padding)


%
% remove paragraph indention
%
\setlength{\parindent}{0pt}
\setlength{\parskip}{6pt plus 2pt minus 1pt}
\setlength{\emergencystretch}{3em}  % prevent overfull lines

%
%
% Listings
%
%


%
% header and footer
%
\usepackage[headsepline,footsepline]{scrlayer-scrpage}

\newpairofpagestyles{eisvogel-header-footer}{
  \clearpairofpagestyles
  \ihead*{MMB8052 - Practical 1}
  \chead*{}
  \ohead*{}
  \ifoot*{Simon J Cockell}
  \cfoot*{}
  \ofoot*{\thepage}
  \addtokomafont{pageheadfoot}{\upshape}
}
\pagestyle{eisvogel-header-footer}



%%
%% end added
%%

\begin{document}

%%
%% begin titlepage
%%
\begin{titlepage}
\newgeometry{top=2cm, right=4cm, bottom=3cm, left=4cm}
\tikz[remember picture,overlay] \node[inner sep=0pt] at (current page.center){\includegraphics[width=\paperwidth,height=\paperheight]{media/background.pdf}};
\newcommand{\colorRule}[3][black]{\textcolor[HTML]{#1}{\rule{#2}{#3}}}
\begin{flushleft}
\noindent
\\[-1em]
\color[HTML]{218ed0}
\makebox[0pt][l]{\colorRule[218ed0]{1.3\textwidth}{4pt}}
\par
\noindent

% The titlepage with a background image has other text spacing and text size
{
  \setstretch{2}
  \vfill
  \vskip -8em
  \noindent {\huge \textbf{\textsf{MMB8052 - Practical 1}}}
    \vskip 1em
  {\Large \textsf{Introduction to the Linux Command Line}}
    \vskip 2em
  \noindent {\Large \textsf{Simon J Cockell} \vskip 0.6em \textsf{}}
  \vfill
}


\end{flushleft}
\end{titlepage}
\restoregeometry
\pagenumbering{arabic} 

%%
%% end titlepage
%%



{
\setcounter{tocdepth}{3}
\tableofcontents
\newpage
}
\hypertarget{introduction-to-the-module}{%
\section{Introduction to the Module}\label{introduction-to-the-module}}

\hypertarget{module-organisation}{%
\subsection{Module Organisation}\label{module-organisation}}

Welcome to \emph{MMB8052 - Bioinformatics for Biomedical Scientists}.
This module is mostly practical in nature - 10 computer lab sessions
will introduce you to the fundamental computing tools used throughout
much of modern bioinformatics, and will also provide case studies of the
\enquote{read world} use of these tools. The lectures in the module will
be delivered by a range of scientists from across the Faculty of Medical
Sciences, and will highlight the \emph{application} of bioinformatics in
modern biomedical research.

\hypertarget{module-assessment}{%
\subsection{Module Assessment}\label{module-assessment}}

Due to the practical emphasis of the module, the assessment will also
focus on these practical aspects. The module is assessed through two
exercises - one short answer quiz in which the solutions to the posed
problems should be derived computationally (more on this later). The
second assessment will be a lab report style write-up of the application
of some of the tools you will learn how to use in the practicals.
Assessment 1 is worth 25\% of the module mark. Assessment 2 makes up the
remaining 75\%.

In addition to these assessed, summative exercises, there will also be a
range of formative tests throughout the module. These will mostly take
the form of multiple choice quizzes, integrated into the practical
sessions or provided separately on Canvas. While these components are
not assessed, or compulsory to complete, they will aid and reinforce
your understanding of the material presented.

\hypertarget{about-this-handbook}{%
\subsection{About This Handbook}\label{about-this-handbook}}

The practical sessions will each be accompanied by a handbook like this
one. These handbooks will provide you with a lot of background
information relevant to the practical at hand, and will provide you with
walk-through instructions for what you are supposed to do in each
session. Computer code (usually intended to be typed in to the
appropriate computational interface - we'll get to what this means
later) will be presented in chunks styled like this:

\begin{Shaded}
\begin{Highlighting}[]
\CommentTok{# this are some command line instructions:}
\NormalTok{$ }\BuiltInTok{pwd}
\ExtensionTok{/home/student}
\NormalTok{$ }\FunctionTok{ls}\NormalTok{ -l -h ~}
\end{Highlighting}
\end{Shaded}

If you see this character: \&\#21AA; in one of these listings, it means
that the command should be entered on one line (the line has been broken
in the listing for presentation reasons only).

Exercises, which will direct you to accomplish some computational task,
will be presented like this:

\begin{longtable}[]{@{}ll@{}}
\toprule
\endhead
\begin{minipage}[t]{0.36\columnwidth}\raggedright
\includegraphics[width=1.5625in,height=\textheight]{media/programming.png}

Estimated time: 2 mins\strut
\end{minipage} & \begin{minipage}[t]{0.58\columnwidth}\raggedright
\textbf{Exercise 1}

The instructions to follow will be in this block of text.

\begin{itemize}
\tightlist
\item
  First instruction
\item
  Second instruction
\item
  etc.
\end{itemize}\strut
\end{minipage}\tabularnewline
\bottomrule
\end{longtable}

\hypertarget{linux}{%
\section{Linux}\label{linux}}

Linux is an umbrella term which describes a family of open-source,
\enquote{Unix-like} computer operating systems which are based on the
Linux kernel. The first Linux operating systems were released in the
mid-90s and today they are used throughout computing, particularly in
the infrastructure which runs the World-Wide Web, and in scientific
computing and virtually all supercomputers. The Android smartphone
operating system is a Linux system. Popular Linux distributions:

\begin{itemize}
\tightlist
\item
  \href{https://ubuntu.com/}{Ubuntu}
\item
  \href{https://getfedora.org/}{Fedora}
\item
  \href{https://www.debian.org/}{Debian}
\item
  \href{https://mxlinux.org/}{MX Linux}
\end{itemize}

\hypertarget{the-kernel}{%
\subsection{The Kernel}\label{the-kernel}}

A kernel is the computer program at the heart of an operating system
(OS), and is responsible for the control of everything in the system.
The kernel facilitates interactions between hardware and software (via
\emph{drivers}) and optimises the use of system resources such as CPU
time and RAM usage. The main kernels in use in modern computing are the
Linux kernel, the Windows NT kernel and the MacOS kernel. All of the
Linux distributions listed above use the same kernel at their core.

\hypertarget{unix}{%
\subsection{Unix}\label{unix}}

The Unix operating system is ancient, in computing terms. It was
conceived and implemented in 1969 at AT\&T's Bell Labs. It is modular by
design, with a number of robust tools each designed to perform a
limited, well-defined function. A program, known as the Unix
\enquote{Shell} provides a text-based interface to these tools, and
allows the user to combine them in order to perform complex workflows.
Thanks to its efficient and robust nature, this computing paradigm
persists today in the modern, Unix-like operating systems, Linux and
MacOS.

\begin{quote}
This is the Unix philosophy: Write programs that do one thing and do it
well. Write programs to work together. Write programs to handle text
streams, because that is a universal interface.

-- Doug McIlory (Bell Labs)
\end{quote}

\hypertarget{the-shell}{%
\subsection{The Shell}\label{the-shell}}

A Unix shell is a command-line user interface for Unix-like operating
systems. It provides a programmable environment for controlling the OS,
and running executable programs. It is typical for the user of a
Unix-like OS to interact with the shell via a \emph{terminal emulator} -
a program which simulates the features of a hardware video terminal
interface. Feature-rich terminal emulators are available for all modern
operating systems.

There are many different shell programs, all of which have different
features. The shells you will encounter most often are bash (the
\enquote{Bourne-Again Shell}) which is the default in most major Linux
distributions, and zsh (Z Shell) which is the default in MacOS. Zsh is
backwards compatible with bash (so any code written for bash will work
in zsh, but not necessarily vice versa).

\hypertarget{logging-in-to-a-linux-server}{%
\subsection{Logging in to a Linux
Server}\label{logging-in-to-a-linux-server}}

Rather than install Linux directly onto your desktops, we will use a
terminal emulation program to log into a virtual cloud server which has
been configured for these practicals. It should be noted that you'll be
using your computer as a dumb terminal (client) that will display the
information generated by programs running on the cloud VM (this is an
example of a \emph{client-server} model).

You should have received an email from Microsoft Azure, inviting you to
register for the lab - click the \enquote{Register for the lab} button
in the email, and log in to Azure Labs with your University credentials.

Once logged in, you should see a heading \enquote{My virtual machines},
with a single entry underneath. Click the slider in the bottom-left of
the VM box to power the machine on. Once it's started up (this will take
a couple of minutes), click the icon in the bottom-right of the VM box
which looks like a small grey monitor. Pick \enquote{Connect via SSH}
from the menu which appears, and set your password when prompted. Once
the password has been set (again, this takes a couple of minutes) select
\enquote{Connect via SSH} again, copy the text in the popup box which
appears - this is your SSH invocation to log in to your VM. For example:

\begin{Shaded}
\begin{Highlighting}[]
\FunctionTok{ssh}\NormalTok{ -p 65432 student@ml-lab-77568ef7-c936-416a-a101-5e2874043ea1.uksouth.cloudapp.azure.com}
\end{Highlighting}
\end{Shaded}

\hypertarget{logging-in-from-windows}{%
\subsubsection{Logging in from Windows}\label{logging-in-from-windows}}

\begin{longtable}[]{@{}ll@{}}
\toprule
\endhead
\begin{minipage}[t]{0.36\columnwidth}\raggedright
\includegraphics[width=1.5625in,height=\textheight]{media/programming.png}

Estimated time: 2 mins\strut
\end{minipage} & \begin{minipage}[t]{0.58\columnwidth}\raggedright
\textbf{Exercise 1a}

\begin{itemize}
\tightlist
\item
  In the \enquote{Port} box, type the number given in the \texttt{-p}
  argument of your SSH invocation (this is likely to be a large number,
  greater than 60000 - from the example above \texttt{65432})
\end{itemize}

PuTTY will then open and prompt you for a password. Enter the password
you set when starting your VM for the first time.\strut
\end{minipage}\tabularnewline
\bottomrule
\end{longtable}

\hypertarget{logging-in-from-macos}{%
\subsubsection{Logging in from MacOS}\label{logging-in-from-macos}}

\begin{longtable}[]{@{}ll@{}}
\toprule
\endhead
\begin{minipage}[t]{0.36\columnwidth}\raggedright
\includegraphics[width=1.5625in,height=\textheight]{media/programming.png}

Estimated time: 2 mins\strut
\end{minipage} & \begin{minipage}[t]{0.58\columnwidth}\raggedright
\textbf{Exercise 1b}

\begin{itemize}
\tightlist
\item
  Open \enquote{Terminal.app} (located in
  /Applications/Utilities/Terminal.app).
\item
  Paste in the connection string shown when you click \enquote{Connect
  via SSH} on the Azure Lab (above). As per the example above:
\end{itemize}

\begin{Shaded}
\begin{Highlighting}[]
\FunctionTok{ssh}\NormalTok{ -p 65432 student@ml-lab-77568ef7-c936-416a-a101-5e2874043ea1.uksouth.cloudapp.azure.com}
\end{Highlighting}
\end{Shaded}

\begin{itemize}
\tightlist
\item
  Enter your password when prompted.
\end{itemize}\strut
\end{minipage}\tabularnewline
\bottomrule
\end{longtable}

\hypertarget{the-linux-filesystem}{%
\section{The Linux Filesystem}\label{the-linux-filesystem}}

\end{document}
